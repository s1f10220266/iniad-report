\documentclass[bachelor]{INIAD}%卒論用 
\addtolength{\footskip}{8mm}
\bibliographystyle{jplain} 
%\usepackage[dviout]{graphicx}
\usepackage[dvipdfmx]{graphicx}
\usepackage{bm}
\usepackage{amsmath}
\usepackage{ascmac}

%\usepackage{geometry}
%\geometry{left=30mm,right=30mm,top=35mm,bottom=30mm}

%\documentclass[oneside]{suribt}% 本文が * ページ以下のときに (掲示に注意)
\title{象の卵生についての研究}
%\titlewidth{}% タイトル幅 (指定するときは単位つきで)
\author{東洋 太郎}
\eauthor{Taro Toyo}% Copyright 表示で使われる
\studentid{1F99999999}
\supervisor{赤羽台 花子 教授}% 1つの引数をとる (役職まで含めて書く)
%\supervisor{指導教員名 役職 \and 指導教員名 役職}% 複数教員の場合,\and でつなげる
\handin{2021}{1}% 提出月. 2 つ (年, 月) 引数をとる
%\keywords{キーワード1, キーワード2} % 概要の下に表示される
\renewcommand{\baselinestretch}{1.25}
\setcounter{tocdepth}{2}

\begin{document}
\mojiparline{40}
\maketitle%%%%%%%%%%%%%%%%%%% タイトル %%%%

\frontmatter% ここから前文

%\etitle{Title in English}

%\begin{eabstract}%%%%%%%%%%%%% 概要 %%%%%%%%
% 300 words abstract in English should be written here. 
%\end{eabstract}

\begin{abstract}%%%%%%%%%%%%% 概要 %%%%%%%%
 ここには論文要旨を記述します。論文要旨の書き方については、指導教員の指導を受けること。
\end{abstract}

%%%%%%%%%%%%% 目次 %%%%%%%%
{\makeatletter
\let\ps@jpl@in\ps@empty
\makeatother
\pagestyle{empty}
\tableofcontents
\clearpage}

\mainmatter% ここから本文 %%% 本文 %%%%%%%%

\include{01_intro}     % はじめに
\include{02_related}   % 関連研究
\include{03_proposed}  % 提案手法

% 以降、実装や評価、結論などの章を適切に配置してください

\backmatter% ここから後付
\chapter{謝辞}
本研究の遂行にあたり、アンケートへの回答にご協力いただいた皆様に感謝いたします。
           % 謝辞

\bibliography{thesis.bib}  % 参考文献

\appendix% ここから付録 %%%%% 付録 %%%%%%%
\chapter{その他の \LaTeX の機能についての紹介}
この章では、本編に含まれない \LaTeX の機能について、その使い方の一部をサンプルとして紹介する。

\section{コード例}
コードを図表に含める場合の例を図\ref{eval_code}に示す。

ここではそのまま利用できる \verb|verbatim| 環境を利用しているが、実際は \verb|jlisting| 等のパッケージを利用したほうがよい
\footnote{適切に設定すれば、行番号やきれいな枠をつけることができる}。

\begin{figure}[tb]
  \begin{screen}
    \begin{verbatim}int main(int argc, char* argv)
{
    printf("hello\n");
    return 0;
}\end{verbatim}
  \end{screen}
  \caption{評価に使用したコード例}
  \label{eval_code}
\end{figure}

\section{様々な箇条書き}
\LaTeX では様々な箇条書きを利用できる。

HTMLの \verb|ul| 要素に相当する順序なしの箇条書きには、\verb|itemize|環境を用いる。

\begin{itemize}
  \item りんご
  \item みかん
  \item バナナ
\end{itemize}

HTMLの \verb|ol| 要素に相当する順序ありの箇条書きには、\verb|enumerate|環境を用いる。

\begin{enumerate}
  \item 起
  \item 承
  \item 転
  \item 結
\end{enumerate}

HTMLの \verb|dl| 要素に相当する説明リストには、\verb|description|環境を用いる。

\begin{description}
  \item[麻婆丼] 500円
  \item[香港風カレー] 500円
  \item[マンゴープリン] 300円
\end{description}

\section{太字やイタリック}
フォントを指定する場合には \verb|textbf| や \verb|textit| 等の命令を用いる。

たとえば、\textbf{太字}にしたり、\textit{斜体}を指定することができる\footnote{ただし、日本語フォントの場合は指定通りの太字や斜体にならない場合もある}。
      % 付録

\end{document}
